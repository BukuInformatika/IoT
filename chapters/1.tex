\section{Internet}
Internet adalah suatu jaringan komputer yang dimana satu jaringan dengan yang lain saling terhubung untuk keperluan komunikasi dan informasi atau dapat disimpulkan internet dapat menghubungkan suatu media elektronik dengan media lainya.

Pada era globalisasi saat ini merupakan salah satu dampak perkembangan dalam bidang Teknologi Informasi(TI). Dengan adanya internet ini segala bentuk informasi menjadi semakin terbuka dan maju.

\subsection{Sejarah Internet}
Pada tahun 1989 internet mulai dikenal di beberapa negara dan mengawali kegiatan secara online.
Penelitian mengenai perangkat yang dikendalikan melalui internet dilakukan John Romkey pada tahun 1990
dengan menciptakan  pemanggang roti yang dapat diaktifkan dan dimatikan secara online.

Selanjutnya berbagai penelitian  perangkat  keras  dan  lunak dilakukan untuk pengendalian jarak jauh melalui internet.
Kevin Ashton, seorang Direktur Eksekutif Auto-ID Lab di MIT menyebutkan pertama kali istilah The Internet of Things (IoT)
pada tahun 1997 berbasis Radio Frequency  Identification (RFID). Selanjutnya  RFID digunakan dalam skala besar di militer
Amerika Serikat sejak tahun 2003 \cite{wilianto2018sejarah}.

\section{Protokol}
Dalam jaringan komputer, protokol merupakan suatu perangkat aturan yang menata atau mengatur komunikasi antar beberapa komputer dalam sebuah jaringan sehingga komputer local dan komputer pada jaringan yang berbeda platform dapat saling mengirimkan informasi dan saling berkomunikasi.

Pada dasarnya protokol merupakan suatu aturan yang mendefinisikan sebuah fungsi seperti mengirimkan pesan, mengirimkan data, mengirimkan informasi dan berbagai macam fungsi lainnya. Fungsi-fungsi tersebut harus dapat dipenuhi oleh pengirim dan penerima supaya komunikasi yang dilakukan berlangsung secara baik dan benar, meskipun sistem yang terdapat dalam jaringan tersebut berbeda. Protokol menangani semua hal yang berkaitan dalam komunikasi data, dari pertukaran data yang memiliki perbedaan format data hingga sampai ke masalah koneksi listrik dalam suatu jaringan. Pada jaringan komputer, terdapat sebuah proses komunikasi dari suatu entiti atau perangkat dengan entiti lainnya yang memiliki sistem berbeda. Entiti tersebut merupakan segala sesuatu yang dapat melakukan proses mengirim dan menerima sehingga dibutuhkan pengertian yang baik antara kedua entiti tersebut agar dapat saling berkomunikasi dengan baik, sama halnya dengan protokol.

\subsection{Jenis-Jenis Protokol}
\begin{enumerate}
\item \textbf{Protokol Ethernet}

Protokol Ethernet merupakan protokol yang sering digunakan pada saat ini. Metode akses yang digunakan oleh Ethernet biasa disebut dengan CSMA(\textit{Carrier Sense Multiple Access}) atau CD(\textit{Collision Detection}. Cara kerja dari protokol Ethernet dimana sebuah sistem pada setiap komputer menunggu intruksi melalui rangkaian kabel sebelum mengirimkan data atau informasi melalui sebuah jaringan. Jika jaringan tidak sibuk, komputer akan mengirimkan informasi data namun jika suatu node lain sudah menyampaikan pesan melalui kabel tersebut, maka komputer akan menunggu dan mencobanya kembali setelah rute telah aman.
\item \textbf{TCP/IP}

TCP/IP atau \textit{Transmission Control Protocol}) atau \textit{Internet Protocol} merupakan standar dari komunikasi data yang dipakai oleh jaringan internet dalam proses tukar-menukar data atau informasi dari satu komputer menuju komputer lainnya dalam jaringan internet.
\item \textbf{UDP}

UDP (\textit{User Datagram Protocol}) merupakan salah satu protokol transpor TCP/IP yang dapat mendukung suatu komunikasi yang unreiable tanpa melalui koneksi antar host dalam suatu jaringan yang menggunakan TCP/IP.
\item \textbf{RTP}

RTP (\textit{Real Time Protocol}) merupakan protokol yang dirancang untuk menyediakan fungsi-fungsi transport jaringan untuk mengirimkan data secara realtime seperti data audio, video melalu layanan multicast atau layana unicast.
\item \textbf{FTP}

FTP (\textit{File Transfer Protocol}) merupakan jenis protokol yang digunakan untuk melakukan upload ataupun mendownload file dimana keamanannya dibuat berdasarkan dari username dan juga password.
\item \textbf{HTTP}

HTTP (\textit{Hyper Text Transfer Protocol}) merupakan protokol yang digunakan untuk melakukan transfer halaman web.
\item \textbf{DHCP}

DHCP adalah singkatan dari (\textit{Dynamic Host Configuration Protocol}), dimana dipakai untuk mempermudah pengalokasian IP Address pada satu jaringan, bila dimana ada jaringan lokal yang tidak menggunakan DHCP maka harus memberikan IP secara manual.
\item \textbf{DNS}

DNS adalah singkatan dari (\textit{Domain Name System}), dimana sebuah DNS adalah standar teknologi yang mengatur penamaan publik dari sebuah website atau dapat juga disebut dengan sebuah sistem yang dapat  menyimpan informasi tentang name host atau nama domain dalam bentuk distributed database didalam jaringan komputer
\item \textbf{ICMP}

ICMP adalah singkatan dari (\textit{Internet Control Message Protocol}), dimana Protokol ICMP ini memiliki tujuan yang berbeda dengan TCP dan UDP dalam hal ICMP tidak digunakan secara langsung oleh aplikasi jaringan milik pengguna.
\item \textbf{IMAP}

IMAP adalah singkatan (\textit{Internet Message Access Protocol}), dimana IMAP adalah protokol standar untuk mengakses atau mengambil e-mail yang berasal dari server atau  IMAP adalah merupakan protokol komunikasi dua arah sebagai perubahan yang dibuat pada local mail yang dikirimkan ke server dan memungkinkan pemakainya untuk memilih pesan e-mail yang akan dia ambil,
\item \textbf{HTTPS}

HTTPS adalah (\textit{Hyper Text Transfer Protocol Secure}) dimana HTTPS itu adalah  bentuk protokol umum, yang sering digunakan untuk mengakses sebuah halaman web, HTTPS dapat kita diartikan sebagai bentuk protokol internet yang paling valid dan yang paling aman. HTTPS ini akan melindungi integritas serta kerahasiaan antara situs dan komputer pengguna.
\item \textbf{SSH}

SSH Yaitu singkatan dari (\textit{Sucure Shell}), SSH adalah protokol jaringan yang memungkinkan pertukaran data secara aman antara 2 komputer. Protokol SSH ini dapat digunakan untuk mengendalikan komputer secara jarak jauh untuk mengirimkan file. Protokol ini  memiliki beberapa kelebihan jika dibandingkan denga protokol yang sejenis seperti FTP, TALNET, DANRSH
\item \textbf{ SSL}

SSL atau singkatan dari (\textit{Secure Socket Layer}), adalah suatu protokol keamanan data yang dipakai untuk menjaga pengiriman data web server dan  pengguna situs web itu sendiri. Dimana SSL adalah standar industri untuk komunikasi web yang aman dan digunakan untuk melindungi jutaan transaksi online setiap hari.
\end{enumerate}



