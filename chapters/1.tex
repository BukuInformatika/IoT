\section{Internet}
Internet adalah suatu jaringan komputer yang dimana satu jaringan dengan yang lain saling terhubung untuk keperluan komunikasi dan informasi atau dapat disimpulkan internet dapat menghubungkan suatu media elektronik dengan media lainya.

Pada era globalisasi saat ini merupakan salah satu dampak perkembangan dalam bidang Teknologi Informasi(TI). Dengan adanya internet ini segala bentuk informasi menjadi semakin terbuka dan maju.

\subsection{Sejarah Internet}
Pada tahun 1989 internet mulai dikenal di beberapa negara dan mengawali kegiatan secara online.
Penelitian mengenai perangkat yang dikendalikan melalui internet dilakukan John Romkey pada tahun 1990 
dengan menciptakan  pemanggang roti yang dapat diaktifkan dan dimatikan secara online.

Selanjutnya berbagai penelitian  perangkat  keras  dan  lunak dilakukan untuk pengendalian jarak jauh melalui internet.
Kevin Ashton, seorang Direktur Eksekutif Auto-ID Lab di MIT menyebutkan pertama kali istilah The Internet of Things (IoT)
pada tahun 1997 berbasis Radio Frequency  Identification (RFID). Selanjutnya  RFID digunakan dalam skala besar di militer
Amerika Serikat sejak tahun 2003 \cite{wilianto2018sejarah}.

\section{Protokol}
Dalam jaringan komputer, protokol merupakan suatu perangkat aturan yang menata atau mengatur komunikasi antar beberapa komputer dalam sebuah jaringan sehingga komputer local dan komputer pada jaringan yang berbeda platform dapat saling mengirimkan informasi dan saling berkomunikasi.   

Pada dasarnya protokol merupakan suatu aturan yang mendefinisikan sebuah fungsi seperti mengirimkan pesan, mengirimkan data, mengirimkan informasi dan berbagai macam fungsi lainnya. Fungsi-fungsi tersebut harus dapat dipenuhi oleh pengirim dan penerima supaya komunikasi yang dilakukan berlangsung secara baik dan benar, meskipun sistem yang terdapat dalam jaringan tersebut berbeda. Protokol menangani semua hal yang berkaitan dalam komunikasi data, dari pertukaran data yang memiliki perbedaan format data hingga sampai ke masalah koneksi listrik dalam suatu jaringan. Pada jaringan komputer, terdapat sebuah proses komunikasi dari suatu entiti atau perangkat dengan entiti lainnya yang memiliki sistem berbeda. Entiti tersebut merupakan segala sesuatu yang dapat melakukan proses mengirim dan menerima sehingga dibutuhkan pengertian yang baik antara kedua entiti tersebut agar dapat saling berkomunikasi dengan baik, sama halnya dengan protokol.

Protokol dapat mendefinisikan apa yang dikomunikasikan bagaimana dan kapan terjadinya suatu komunikasi. Untuk menjalakan sebuah komunikasi yang baik dan benar, dibutuhkan beberapa elemen-elemen penting dalam suatu protokol. Elemen-elemen tersebut adalah :
\begin{itemize}
\item \textbf{Syntax} 
\item \textbf{Semantics} 
\item \textbf{Timing} 
\end{itemize}

 
